\documentclass[a4paper,11pt,oneside,openany,uplatex]{jsbook}

%図表の個数などの設定.
\setcounter{topnumber}{4}
\setcounter{bottomnumber}{4}
\setcounter{totalnumber}{4}
\setcounter{dbltopnumber}{3}

\renewcommand{\topfraction}{.95}
\renewcommand{\bottomfraction}{.90}
\renewcommand{\textfraction}{.05}
\renewcommand{\floatpagefraction}{.95}

%使用するパッケージを記述.
\usepackage{amsmath, amssymb} %複雑な数式などを打つときに使用.
\usepackage{bm} %数式環境内で太字を使うときに便利.
\usepackage{graphicx} %画像を挿入したり,テキストや図の拡大縮小・回転を行う.
\usepackage{subfigure} %図を並べる(今はsubfigとかsubcaptionとかが推奨らしい.よく知らない)
\usepackage{verbatim} %入力どおりの出力を行う.
\usepackage{ascmac} %テキストを枠で囲んだりできるが,微小なズレがでたりする.
\usepackage{makeidx} %索引を作成できる.
\usepackage{dcolumn} %表の数値を小数点で桁を揃える.
\usepackage{lscape} %図表を90度横に倒して配置する.
\usepackage{setspace} %行間調整.

%余白の設定.
\setlength{\textwidth}{150truemm}      % テキスト幅: 210-(30+30)=150mm
\setlength{\fullwidth}{\textwidth}     % ページ全体の幅
\setlength{\oddsidemargin}{30truemm}   % 左余白
\addtolength{\oddsidemargin}{-1truein} % 左位置デフォルトから-1inch
\setlength{\topmargin}{15truemm}       % 上余白
\setlength{\textheight}{242truemm}     % テキスト高さ: 297-(25+30)=242mm
\addtolength{\topmargin}{-1truein}     % 上位置デフォルトから-1inch


\renewcommand{\labelenumi}{(\arabic{enumi}) } %enumerate環境を1. 2.の形式から(1) (2)の形式へ変更(文書全体).

%\setcounter{tocdepth}{2} %項レベルまで目次に反映させるコマンド.

\begin{document}


\begin{titlepage}
\begin{spacing}{2.3}

\begin{center}
\vspace*{80truept}
{\huge 人文社会系研究科}\\
\vspace{30truept}
{\huge 修 \ 士 \ 学 \ 位 \ 論 \ 文}\\
\vspace{30truept}
{\huge タイトル}\\ % タイトル
{\LARGE ------サブタイトル------}\\ % サブタイトル(なければコメントアウト)
\vspace{100truept}
{\LARGE xxxx年12月 \ 提出}\\ % 提出日
{\LARGE 社会学専門分野}\\ %所属
{\LARGE 12-345678}\\ % 学籍番号
{\LARGE 麦山 亮太}\\ % 著者
\end{center}

\end{spacing}
\end{titlepage}

%
 %「表紙.tex」というファイルを同じディレクトリ上においておくこと.
%
\frontmatter

\tableofcontents

 % 本文
 %本文は章や節ごとに分割して別々のファイルにして管理し,\input{}で挿入していくのがよいだろう.
\mainmatter

\chapter{はじめに}
\section{さらにはじめに}
\subsection{さらにさらにはじめに}

%あとがき
 \backmatter
\chapter{あとがき}
ありがとうございました.

%参考文献(bibtex不使用で,手動で入力する場合.字下げの形式は『社会学評論スタイルガイド』にしたがっている)
\def\bibindent{1.85em}
\begin{thebibliography}{99\kern\bibindent}
\makeatletter
\def\@biblabel#1{}
\let\old@bibitem\bibitem
\def\bibitem#1{\old@bibitem{#1}\leavevmode\kern-\bibindent}
\makeatother
\small
\bibitem{}
Granovetter, Mark. 1973. ``The Strength of Weak Ties." \textit{American Journal of Sociology} 78(6): 1360--80.
\end{thebibliography}
\end{document}